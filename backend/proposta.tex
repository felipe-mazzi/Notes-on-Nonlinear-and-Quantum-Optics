\chapter*{Proposta}

Para evitar que eu mesmo me perca do objetivo original dessas anotações, segue uma exposição da lógica que está me orientando aqui.

Apenas ler ou ouvir os conteúdos nunca foi suficiente para que eu aprendesse, ou sequer mantivesse a atenção presa por tempo suficiente. Além disso, é muito difícil retomar conteúdos estudados sem boas anotações. Tudo isso se torna ainda mais importante quando o estudo não é voltado para uma prova, e sim para uma boa formação geral da área de interesse. Portanto, entendo que é impensável pensar em um plano de estudos aprofundados, de longo prazo, e sobre um tema amplo e complexo sem anotações.

Todos esses benefícios têm um custo. Fazer anotações pode ser extremamente trabalhoso, e sem ter os princípios certos em mente, o hábito pode facilmente se tornar muito improdutivo. É preciso ter clareza sobre o que é o grau adequado de preocupação com detalhes, padronização, originalidade e compreensibilidade para outros.

Se um raciocínio ou demonstração é óbvio para mim, ou se pode ser obtido facilmente a partir de um método usual, não é preciso anotá-lo. Isso vale para a solução de equações diferenciais, para a definição de quantidades físicas ou para a escrita de igualdades. Anotações não são um projeto de livro didático.

Padronização é útil para que se gaste menos tempo pensando sobre a estrutura do texto. Não vou me preocupar com isso aqui, usarei apenas o bom senso. Isso vale para a notação matemática, a hierarquia das subdivisões e para o grau de aprofundamento em cada seção.

Também não vou me preocupar em ser original. Vou copiar figuras e trechos inteiros de texto com bastante frequência, sem me preocupar em dar os devidos créditos, salvo quando for útil para referência futura. Farei essas anotações pensando sempre que eu serei o único leitor.

Por último, quando estiver escrevendo minhas próprias impressões, não farei nenhum esforço para ser conciso. Um dos maiores benefícios de fazer anotações em um computador é a facilidade de escrever muito sem se cansar. Gostaria muito de conseguir, ao final de cada capítulo, escrever uma visão geral puramente textual do que foi aprendido, tão longa quanto for necessário.

Posso manter esse sistema por alguns meses, e será interessante ver como essa filosofia mudou.
